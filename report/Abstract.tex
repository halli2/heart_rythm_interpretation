\section*{Abstract}
This project developes a convolutional neural network (CNN) to classify hearth rhythms in five different categories, ventricular fibrillation (VF), ventricular tachycardia (VT), asystole (AS), pulseless electrical activity (PEA), and pulse generating rhythm (PGR). The project was based around earlier projects from the university, and also took inspiration from studies outside of this. The necessary background to understand the project is presented, and a detailed description of both the dataset and the model that was made follows. The main purpose of the experiments was to find hyper parameters that would increase the accuracy of the model, these are presented in the results section. Lastly it was concluded that this model is generally a good fit for classifying hearth rhythms, and some suggestions for further improvement were made. The code and a description of how it works can be found in the git repository hosted at~\href{https://github.com/halli2/heart_rythm_interpretation}{GitHub}.

%Classifying the different heart rhythms of a patient undergoing cardiac arrest can help to decide if the patient should be shocked or not. In this project we are using a convolutional neural network (CNN) to classify the heart rhythm between the categories coarse ventricular fibrillation (VF), rapid ventricular tachycardia (VT), asystole (AS), pulseless electrical activity (PEA) and pulse generating rhythm (PGR). The main experiments are exploring different hyperparameters(HPs) for the CNN. The HPs that are tested is Number of convolutional layer($N$), dropout, kernel size($K$), filter size($F$) and number of nodes in the fully connected hidden layers($FC$). To evaluate our models we used Balanced Accuracy(BAC). In the first set of experiments we found out that the best models was with $N=5$. There where no good indicator for which $K$ or $F$ that was the best. So for the second set of experiments we put some of the HPs to constant values. The HPs that where set was dropout = 0.3, $FC_1=64$ and $FC_2=32$. From this experiments we found the best model to be with $N=4$. This model got tested further to see the robustness of the model. In general this type of structure is a good model for classifying the heart rhythms. For the code and how it works look at the git repository hosted at \href{https://github.com/halli2/heart_rythm_interpretation}{GitHub}.